\documentclass[a4paper,10pt]{article}
\usepackage[utf8]{inputenc}

%opening
\title{}
\author{}

\begin{document}

\maketitle

\begin{abstract}

\end{abstract}

\section{Channel capacity optimization for single subband}
%% Thourthput optimization
% We assume a wideband received signal. Which is noise plus the signal
% transmitted by the other users.
 
% We can not say that the spectrum change abrubptly from one subband to the
% other, and in realit there is a subband to subband interference.
 
%Anyway, if we assume that the subband have a square shape and that we can
%have very long FFTs, then.
 
 For a given subband we can obtain the optimum sensing time which
  maximizes the probability of Pnf = ( 1- Pfa ).
 The channel capacity for the subcahnnel kwill be:
 
 
 $$ C = \frac{T_f}{T_f + T_0}B_k \log_2(1+ \gamma_k ) \cdot P(H_0) \cdot  (1- P_{f}) $$
 
%% For the moment we will not consider the transmission under missdetection assumption.


%%$$ $$
 
 $$ P_f = Q(\lambda) $$
 
 s.t. $$ P_d = Q{ \left( \frac{\lambda - \mu }{\sigma}  \right) } =
 \hat{P_d} $$
 Where:
 
 \begin{equation}
   \mu = \sqrt{ \frac{N_k N_i }{N_i + N_k}  }, \hspace{1cm}  
   \sigma = (1 + \gamma_k)^2,  
   \hspace{1cm} N_k = B_k \ T_0, N_i = \textbf{Constant}   
 \end{equation}
  
The optimal $\lambda$ will be a function of the sensing time $T_0$.


 \begin{equation}
  \lambda        =
 \sigma \  Q_\textrm{inv}   ( \hat{P_d}   )    +  \mu  =  
   \sigma \  Q_\textrm{inv}   ( \hat{P_d}   )    + \sqrt{ \frac{N_k N_i }{N_i + N_k}  }
\end{equation}
 
 
  \begin{equation}
  \lambda        = 
   \sigma \  Q_\textrm{inv}   ( \hat{P_d}   )    + \sqrt{ \frac{   \alpha }{ \alpha   + 1}  } B_k T_0
\end{equation}


with $N_i = \alpha N_k$

Replacing  in the Capacity:

$$ C = \frac{T_f}{T_f + T_0}R_k  \cdot P(H_0) \cdot \left(1-   Q \left(   \sigma \  Q_\textrm{inv}   ( \hat{P_d}   )    + \sqrt{ \frac{   \alpha }{ \alpha   + 1}  } B_k T_0  \right)      \right) $$
 

\begin{equation}
 C' =    T_f \ R_k  \ P(H_0) \cdot 
 \left[ \frac{1}{T_f + T_0} \cdot 
 \left(1-   Q \left(   \sigma \  Q_\textrm{inv}   ( \hat{P_d}   )    + \sqrt{ \frac{   \alpha }{ \alpha   + 1}  } B_k T_0  \right)      \right)  
\right]'
 \end{equation}
  
  
  \begin{equation}
 C' =    T_f \ R_k  \ P(H_0) \cdot 
 \left[ \frac{ \left(1-   Q \left(   \sigma \  Q_\textrm{inv}   ( \hat{P_d}   )    
 + \sqrt{ \frac{   \alpha }{ \alpha   + 1}  } B_k T_0  \right)      \right)}{T_f + T_0}    
\right]'
 \end{equation}

 
   \begin{equation}
 C' =    a_0 \cdot 
 \left[ \frac{  -(Tf + T_0) \cdot 
  Q' \left(   a_1   
 + a_2 T_0  \right)     
 -\left(1-   Q \left(  a_1
 + a_2 T_0  \right)      \right)}
 { (T_f + T_0)^2 }    
\right]
 \end{equation}
 
 
 
 
 
 
 
 
 
\end{document}
